\documentclass[utf8]{article}
\usepackage[utf8]{inputenc}
\usepackage[parfill]{parskip}
\usepackage{amsmath}
\usepackage{amssymb}
\usepackage{amsfonts}
\usepackage{graphicx}
\usepackage{float}
\usepackage{enumitem}
\usepackage{listingsutf8}
\usepackage{fullpage}
\usepackage{fancyheadings}
\usepackage[absolute]{textpos}
\usepackage[english]{babel}
\usepackage[autolanguage]{numprint}
\usepackage{amsmath}
\usepackage{algorithm,algorithmic,amsmath}

\begin{document}
\begin{titlepage}
\begin{textblock*}{5cm}(9mm,9mm)
\includegraphics[scale=0.5]{ULB.jpg}
\end{textblock*}
\author{CHRETIEN Marcus, O'CUILLEANAIN Thomas}
\date{\today}
\title{
    \begin{minipage}\linewidth
        \centering
        Project Report
        \vskip10pt
        \large\textbf{Chatroom}
    \end{minipage}
}
\maketitle
\centering
\begin{figure}[H]
  \centering
	\includegraphics[scale=0.4]{SVT.png}
  \label{fig:logo}
\end{figure}
\end{titlepage}


\tableofcontents
\newpage

\section{Introduction}

The goal of this project is to create a basic chatroom capable of sending timestamped and signed messages over a network. This was done by implementing a basic client, capable of connecting and thus sending said messages to a server, and  a multi-client server which forwards the messages it receives to all clients connected to it.

\section{Choices of implementation}

\subsection{Client}

\subsubsection{Sending and recieving messages}
Once we initialised the connection between the client and the server, we decided to manange the exchange of messages by creating two threads client side, one to manage all reception of messages and the other to manage all sending of messages. \par
This is because the the system calls $read$ and $write$ are blocking calls. This means that without using a thread, a user would be unable to send a message as he is being blocked by $read$ which is waiting for an incoming message.

\subsubsection{Passing more than one argument to a thread}
When sending a message, the function in charge of doing so required the client's $pseudo$ and $socket$, however, when creating a thread using $pthread$\_$create$ you are limited to one argument for the function assigned to the thread.\par
Therefore we implemented a $struct$ that holds all the data required, which was then passed as the argument in said function.

\subsection{rawtime}

\section{Limitations of the project}

\subsection{Bug after copy-pasting 1500 numbers with spaces}

\subsection{Receiving a message whilst in the process of sending one}

\section{Difficulties encountered}

\subsection{using select instead of threads to manage clients}

\subsection{time.h}

\section{Original solutions}

\subsection{Displaying pseudo}

\section{Additional reasearch and testing}

\subsection{Findings when testing limitations}



\end{document}